\documentclass[12pt,a4paper]{ltjsarticle}
\usepackage[hiragino-pro,deluxe]{luatexja-preset}
\usepackage{fancybox}
\usepackage{amsmath}
\newcommand{\jugem}{寿限無 寿限無 五劫の摺り切れ 海砂利水魚の 水行末 雲来末 
  風来末 食う寝る所に住む所, Supercalifragilisticexpialidocious!}
\title{\tt{ltjsarticle}のフォント設定}
\begin{document}
\maketitle
\section{フォントの指定方法}
\label{sec:font}
フォントは「family」「series」「shape」の3つを組合せて指定する.書式は以下の通り
\VerbBox{\ovalbox}{%
\verb+{\**family \**series \**shape 寿限無 寿限無 五劫の摺り切れ…}+
}
\begin{description}
\item[family] 明朝/ゴシックの用にフォントの「体」に相当する.欧文フォントと和
  文フォントでそれぞれ個別に指定できるが,\tt{ltjsarticle}の場合,欧文フォントを指
  定すると,それに追従して和文フォントも変更される.逆に,和文フォントを指定した
  だけでは欧文フォントは追従変更されず,デフォルト(Roman体)のままとなる.
  \begin{table}[h]
    \centering
      欧文フォント\\
    \begin{tabular}{rl}
      \hline
      \tt{\textbackslash{}rmfaimly} & {\rmfamily Roman/明朝体}\\
      \tt{\textbackslash{}sffamily} & {\sffamily Sans-serif/ゴシック体}\\
      \tt{\textbackslash{}ttfamily} & {\ttfamily Typewriter/ゴシック体}\\
      \hline
    \end{tabular}\\
      和文フォント\\
    \begin{tabular}{rl}
      \hline
      \tt{\textbackslash{}mcfaimly} & {\mcfamily 明朝体 (Roman)}\\
      \tt{\textbackslash{}gtfamily} & {\gtfamily ゴシック体 (Roman)}\\
      \tt{\textbackslash{}mgtfamily} & {\mgfamily 丸文字ゴシック体 (Roman)}\\
      \tt{\textbackslash{}gtebfamily} & {\gtebfamily 極太ゴシック体 (Roman)}\\
      \hline
    \end{tabular}
  \end{table}
\item[series] フォントの「太さ」を指定する.普通もしくは太字の2種類のみ.
  \begin{table}[h]
    \centering
    \begin{tabular}{rl}
      \hline
      \tt{\textbackslash{}mdseries} & {\mdseries 標準の太さ (medium)}\\
      \tt{\textbackslash{}bfseries} & {\bfseries 太字 (bold)}\\
      \hline
    \end{tabular}
  \end{table}
\item[shape] フォントの「修飾」を指定する.和文フォントには影響しない.
  \begin{table}[h]
    \centering
    \begin{tabular}{rl}
      \hline
      \tt{\textbackslash{}upshape} & {\upshape 直立体(default)}\\
      \tt{\textbackslash{}itshape} & {\itshape 斜体(italic)}\\
      \tt{\textbackslash{}slshape} & {\slshape 傾斜体(slant).直立体を傾けただけ}\\
      \tt{\textbackslash{}scshape} & {\scshape スモールキャピタル体(small
        capital)}\\
      \hline
    \end{tabular}
  \end{table}
\end{description}
\section{フォント設定の省略形}
family, series, shape を設定するのは面倒なので,既に定義されているものもある.
\begin{description}
\item [\rm Roman/明朝\tt{(\textbackslash{}rm)}] {\rm \jugem}
\item [\bf Roman/明朝・太字\tt{(\textbackslash{}bf)}] {\bf \jugem}
\item [\sf Sans-serif/ゴシック\tt{(\textbackslash{}sf)}] {\sf \jugem}
\item [\tt Typewriter/ゴシック\tt{(\textbackslash{}tt)}] {\tt \jugem}
\item [\it Roman/明朝・イタリック\tt{(\textbackslash{}it)}] {\it \jugem}
\end{description}
\end{document}

%%% Local Variables: 
%%% mode: latex
%%% TeX-master: t
%%% End: 
